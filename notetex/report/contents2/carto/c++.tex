

constexpr是C++11中新增的关键字,其语义是“常量表达式”,也就是在编译期可求值的表达式。

std::numeric\_limits::min() 返回的是编译器定义的最小的双精度浮点数。

std::numeric\_limits::epsilon() 运行编译程序的计算机所能识别的最小非零浮点数。

std::mt19937 相对于传统的srand(),std::mt19937拥有更好的性能。

std::uniform_real_distribution 产生均匀分布的随机数。



Eigen::Matrix<T, 3, 1>::UnitX 


vector 的reserve增加了vector的capacity,但是它的size没有改变!而resize改变了vector的capacity同时也增加了它的size

reserve是容器预留空间,但在空间内不真正创建元素对象,所以在没有添加新的对象之前,不能引用容器内的元素。加入新的元素时,要调用push_back()/insert()函数。

resize是改变容器的大小,且在创建对象,因此,调用这个函数之后,就可以引用容器内的对象了,因此当加入新的元素时,用operator[]操作符,或者用迭代器来引用元素对象。
此时再调用push_back()函数,是加在这个新的空间后面的。

push_back()函数向容器中加入一个临时对象(右值元素)时, 首先会调用构造函数生成这个对象,然后条用拷贝构造函数将这个对象放入容器中, 最后释放临时对象。
但是emplace_back()函数向容器中中加入临时对象, 临时对象原地构造,没有赋值或移动的操作。


std::transform在指定的范围内应用于给定的操作,并将结果存储在指定的另一个范围内。要使用std::transform函数需要包含<algorithm>头文件。

std::back_inserter用于在末尾插入元素。


explicit构造函数:

\begin{itemize}
  \item 将构造函数声明为explicit 来阻止类类型隐式转换
  \item 关键字explicit 只对一个实参的构造函数有效
  \item explicit 构造函数只能用于直接初始化
  \item 在类内声明构造函数时使用explicit 关键字,在类外部定义时不应重复
\end{itemize}

std::forward_as_tuple() 通过内部元素,使用右值引用构造的形式.

decltype: 有时我们希望从表达式的类型推断出要定义的变量类型,但是不想用该表达式的值初始化变量(初始化可以用auto)。为了满足这一需求,
C++11新标准引入了decltype类型说明符,它的作用是选择并返回操作数的数据类型,在此过程中,编译器分析表达式并得到它的类型,却不实际计算表达式的值。

decltype和auto都可以用来推断类型,但是二者有几处明显的差异:

\begin{itemize}
  \item auto忽略顶层const,decltype保留顶层const;
  \item 对引用操作,auto推断出原有类型,decltype推断出引用;
  \item 对解引用操作,auto推断出原有类型,decltype推断出引用;
  \item auto推断时会实际执行,decltype不会执行,只做分析。总之在使用中过程中和const、引用和指针结合时需要特别小心。
\end{itemize}

迭代器的下一个或上一个分别用std::next和std::prev获取.


int t=std::lower_bound(a+l,a+r,m)-a  在升序排列的a数组内二分查找[l,r)区间内的值为m的元素。返回m在数组中的下标。


GUARDED_BY(mutex_) : 
guarded_by属性是为了保证线程安全,使用该属性后,线程要使用相应变量,必须先锁定mutex_

for (const Eigen::Vector3f& point : *this) ??

enum class 和 enum struct

\begin{itemize}
  \item enum class 和 enum struct 是等价的
  \item 使用enum class \enum struct不会与现存的enum关键词冲突。而且enum class \enum struct具有更好的类型安全和类似封装的特性(scoped nature)。
  \item 与整形之间不会发生隐式类型转换,但是可以强转。
  \item 默认的底层数据类型是int,用户可以通过:type(冒号+类型)来指定任何整形(除了wchar_t)作为底层数据类型。
  \item 引入了域,要通过域运算符访问,不可以直接通过枚举体成员名来访问
\end{itemize}


