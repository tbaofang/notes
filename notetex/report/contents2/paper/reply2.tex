\section{研究背景}
各位答辩老师,同学,你们好,我是唐宝芳,目前在广东博智林机器人公司担任导航算法工程师职位.
论文研究的题目是XXX \qquad
我将从以下三个部分展开叙述:
首先,是研究的背景,将会讲到机器人概念,及开发过程中遇到问题及挑战;
其次,研究的内容,将详细讲述导航系统的实现过程,包括建图/定位和规划;

首先机器人的概念,根据机器人运用的环境,国际机器人联盟上将机器人分为:工业机器人和服务机器人两类,
我国电子协会考虑到一些特殊应用情况,将机器人划分为工业/服务和特种机器人.服务机器人又分为商业级和消费级.
以下是具体的分类,在过去两年多时间我从事开发过扫地机器人/巡检机器人/AGV机器人,分别属于这里分类中的家政机器人,安防机器人和物流机器人.


在机器人导航算法开发过程中发现存在以下一些问题:
比如:XXX


\section{内容}


针对前面分析的问题,
论文将围绕着解决移动机器人导航系统中的建图/定位和路径规划相关算法及实现展开研究.
论文中我完成了以下三部分内容:
第一个是建图,XXX;\qquad 第二个是定位,XXX; \qquad 第三个是规划,XXX.

\subsection{建图}
论文中建图我采用了两种方法,一个是滤波的方法一个是图优化的方法.
首先是滤波的方法:
我将其实现分成三个步骤,分别是:XXX;\qquad
在传感器数据收集阶段,收集了XXX;\qquad 
在算法实现阶段,我在RBPF算法的基础上采取了提议分布和选择性重采样方法,有效的解决建图过程中需要大量粒子数的问题,同时缓解了粒子退化的问题.
最后,构建出了最右边的栅格地图.

同时,其次是图优化的建图方法,
其实现流程和滤波建图方法类似,不同的是它通过图构建和优化两步来实现建图过程.
图构建也称为建图前端,它是用节点表示机器人的轨迹位姿,两个节点之间的边表示位姿之间的空间约束,约束包含顺序配准约和闭环检测约束.
建图后端的优化就是找到一个最优配置,使得预测与观测的误差最小.

最后对这两种建图方法进行比较:XXX

\subsection{定位}

有关定位研究,它的开发步骤和建图是类似的,
不同的是在在数据采集中多了一个前面建图模块发布的Map,
在数据处理部分我采用了UKF算法和MCL算法的二次定位方法,最后将获得的位姿信息发布出去.

定位的数据处理部分具体实现过程是:通过UKF融合Odom和IMU数据获得ukf\_pose,
然后将该pose作为MCL的预测值,同时融合激光的扫描数据与地图匹配的结果,获得最终的二次定位结果.

当时开发选择使用UKF的原因是:XXX \qquad
而选择MCL的原因是:XXX

\subsection{规划}

最后一部分路径规划研究,也是自动导航系统功能的体现部分:
规划包括了全局路径规划和局部路径规划
全局路径规划中我开发了A*算法:它包含了两个步骤,栅格代价值计算和路径搜索,最后获得左图从起点到达目标点的
% 左图灰色部分表示了搜索的范围;(每个栅格的代价值,颜色越深代价越高),
% 通过步骤二的搜索就能找到从起始点到目标点的最短路径.
接下来局部规划部分:局部规划使用了一种叫DWA的算法,有三个步骤分别是:速度采样/轨迹生成和轨迹评分,最后DWA算法的最终输出的是机器人的控制指令.

规划里面除了包含全局规划和局部规划还我还开发了Costmap,它为全局规划和局部规划提供不同的代价地图.\qquad
XXX

接下来是全局规划,XXX

最后一个是与自主导航直接相关的模块,局部规划器:XXX

这个框图就包含了整个导航系统关键模块,在方框的外部,包含了一开始研究的建图,和定位,还包含实时的传感器感知数据.
方框的内部,就是规划部分,包括代价地图,全局规划和局部规划,
最后通过设置一个目标点,机器人就能自主移动到目标点.


% 滤波方法无法有效消除累计误差问题,而图优化方法通过闭环检测能有有效消除机器人运行过程中的累计误差.
% 图优化方法将建图分成图构建和优化两步.
% 构建部分是将

% 我在RBPF算法的基础上改进了



\section{总结与展望}

我是在三家不同公司开发了导航系统的建图/定位和规划模块.而在同一家公司这些模块一般由不同的工程师负责.

现在及将来会继续研究的功能算法

我的论文讲解到此结束.再次感谢各位到老师和同学.

强化学习是一个理论框架,它解决的是一类有目标导向的交互式学习问题.包含三个基本要素,分别是智能体与环境交互过程中的状态/动作/奖励(收益)
在导航系统中的定位/规划/避障算法都可以归类到这个交互式学习问题中.
