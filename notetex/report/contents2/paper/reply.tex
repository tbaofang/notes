\section{背景}
各位答辩老师,同学,你们好,我是唐宝芳,目前在广东博智林机器人公司担任导航算法工程师职位.
本次课题研究的题目是:室内地面移动机器人自主导航系统关键算法及实现.
我将从以下三个部分讲述:
首先,是研究的背景,将会提到从机器人分类,遇到问题.
其次,研究的内容,将详细讲述导航系统的实现过程,包括建图/定位和规划.
最后,进行研究总结与展望.

首先我们通过分类来认识一下机器人,国际上将机器人分为:工业机器人和服务机器人两类,
我国电子协会考虑到特殊情况,将机器人划分为工业/服务和特种机器人.服务机器人包括商业级和消费级.
在过去两年多时间我从事开发过扫地机器人/巡检机器人/AGV机器人,分别属于这里划分的家政机器人,安防机器人和物流机器人.


在开发过程中发现机器人存在以下一些问题:
比如:地图构建方面,大规模场景地图构建存在存储空间和计算量不足,建图过程中易受动态环境干扰,建图的准确性和精度不高等问题.
在自主定位方面,会随着应用场景的复杂性和动态性增加,利用激光扫描数据做匹配算法存在效率低的问题;还有当遇到打滑或长走廊情况,
容易出现定位偏差过大甚至定位失败的问题.
在路径规划方面,其中全局规划算法存在计算量大耗时长,而局部规划存在避开障碍物不够灵活或者有时避障失败等问题.


\section{内容}
\subsection{概述}
针对前面分析的问题,
论文将围绕移动机器人导航系统中的建图/定位和路径规划相关算法及实现展开研究.
建图:构建出一张机器人能够理解的环境地图模型.
% (构建一张地图,这张地图是机器人能够理解的,比如论文中采样的栅格地图,它用三种状态表示了环境,这三种状态分别是:占据/空闲/未知).
定位:确定机器人在地图中的位姿信息.
规划:在地图中寻找一条从起始点到目标点的无碰撞路径.

右图可以看出建图定位规划功能模块之间并不是独立存在的,而是具有一定的关联性.

% 右图是建图/定位/规划与环境之间依存关系,可以发现建图定位规划并不是独立的功能模块,而是存在相互依赖的关系.

% 这张关系图中我们可以看到地图构建/自主定位/路径规划并不是独立存在的功能,而是存在相互依赖的关系.
% 定位和规划都依赖于地图,同时建图也依赖于定位,有一种方法叫做SLAM,能够同时完成了定位和建图的功能.


\subsubsection{建图}

论文中我完成了以下三部分内容:
第一个是建图,通过建议分布/自适应采样策略改进了RBPF算法,使得环境地图构建更加高效,其次研究并采样图优化建图方法.
第二个是定位,采用了多传感器融合技术,并通过UKF和MCL的二次定位方法,解决了机器人出现打滑或进入长走廊出现的定位偏差过大问题.
第三个是规划,开发A*算法实现了全局路径规划,和开发DWA算法实现局部路径规划.

有关滤波方法建图:
关于建图,分成三个步骤,分别是:传感器数据收集/建图算法实现/地图发布
在传感器数据收集阶段,收集了odom数据,IMU数据,雷达扫描数据.
在算法实现阶段,我在RBPF算法的基础上采取了提议分布和选择性重采样方法,有效的解决建图过程中需要大量粒子数的问题,同时缓解了粒子退化的问题.
最后,构建出了最右边的栅格地图.

同时,在建图方面我还研究了基于图优化的建图方法,
其实现流程和滤波建图方法类似,不同的是图优化采用优化的方法.不同的是它通过图构建和优化两步来实现建图过程.
前端图构建是用节点表示机器人的轨迹位姿,两个节点之间的边表示位姿之间的空间约束,包含两类约束一类是顺序配准约束另一类是闭环检测约束.
后端优化就是找到一个最优配置,使得预测与观测的误差最小.

现在对这两种建图方法进行比较:
一般小场景(如:家庭)建图会倾向使用滤波方法,因为它的计算量较小,比较适合低成本的导航方案.
大场景(如:工厂)滤波方法一般需要大量的粒子来获取较好的建图效果,而每个粒子都会包含整个地图信息,从而导致计算量随场景增大成指数增长.
从而不适用于大场景的建图.图优化方法只是随场景增大少量增加,因而更适合大规模场景中的建图.

\subsection{定位}

有关定位研究,它的实现步骤和建图是类似的,
不同的是在在数据采集中多了一个Map,
在数据处理部分我采用了UKF算法和MCL算法同时给机器人定位,最后将获得的位姿发布出去.

定位的数据处理部分具体实现过程是:通过UKF融合Odom和IMU数据获得ukf\_pose,
然后将该pose作为MCL的预测值,同时融合激光的扫描数据与地图匹配的结果,获得最终的二次定位结果.

当时设计选择使用UKF的原因是:
UKF避免了像EKF复杂非线性函数雅可比矩阵的复杂运算;
同时保证了在非线性系统中的适应性.

而选择MCL的原因是:
能够适用于非线性非高斯的系统.
设计了ε-贪心策略采样保证系统拥有足够的鲁棒性.

\subsection{规划}

最后一部分研究路径规划:
规划包括了全局路径规划和局部路径规划
全局路径规划中使用了A*算法:它包含了两个步骤,栅格代价值计算和路径搜索,
左图灰色部分表示了搜索的范围;(每个栅格的代价值,颜色越深代价越高),
通过步骤二的搜索就能找到从起始点到目标点的最短路径.
接下来是局部规划部分:局部规划使用了一种叫DWA的算法,有三个步骤分别是:速度采样/轨迹生成和轨迹评分,最后DWA算法的最终输出的是机器人的控制指令.

规划里面除了包含全局规划和局部规划还有Costmap,它为全局规划和局部规划提供不同的代价地图.
一般有三个图层来表示:分别是静态地图层,障碍物层和膨胀层.

接下来是全局规划,采用基于启发式搜索策略A*搜索算法作为全局规划算法.
代价值计算:从机器人的起始位置开始,逐步向周围节点扩展,直到目标点.
路径搜索:从目标点开始采样梯度下降方法搜索,直到起始位置.得到一条从起始位置到目标位置的最短路径.

接下来的局部规划器:其目标是跟踪全局规划器输出的全局路径,并依据机器人的动态特性和周围障碍物的特征,生成控制机器人运动的速度指令.
实现过程为:速度采样:在机器人的速度空间(线速度/角速度)中离散地采样.
轨迹生成:对每个样本,在机器人当前状态的基础上推演未来很短一段时间内的运动轨迹. 
轨迹评分:根据发生碰撞的可能性/目标点的接近程度/全局轨迹的跟踪近似度/速度限制等多方面的评价指标对这些轨迹打分,
			选取得分最高的轨迹,将其对应的指令下发给底盘,控制机器人运动.



% 滤波方法无法有效消除累计误差问题,而图优化方法通过闭环检测能有有效消除机器人运行过程中的累计误差.
% 图优化方法将建图分成图构建和优化两步.
% 构建部分是将

% 我在RBPF算法的基础上改进了



\section{总结与展望}

开发基于 RBPF 改进的建图算法,并在扫地机器人得到了验证.
开发UKF和MCL的二次融合定位算法,并在巡检机器人得到了验证.
开发全局规划A*算法和局部规划DWA算法,并在AGV机器人得到了验证.

最后的展望,也是我接下来主要研究的工作,
1 建图方面,将继续进一步研究适用于大规模场景建图方案,也就是图优化的方法.
2. 机器人预测方面,研究对行人的识别与跟踪,对人的行为进行预测,来提高人机交互的性能.
3.多机器人导航方法,将研究多机器人协同导航系统,克服单机器人在时间.空间及计算资源上的不足.
