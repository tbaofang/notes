\section{Cartographer}
% \begin{frame}{参数配置}
%   AMCL关于odom的参数配置列表(以差速模型为例):

%   % \frametitle{表格}
%   \begin{table}[htbp!]
%     \centering
%     % \caption{}
%     \begin{tabular}{c|c|c}
%       \toprule[1pt]
%       参数	& 默认值 & 描述  \\
%       \toprule[1pt]
%       odom\_model\_type	& diff & 里程计运动模型,种类有diff、ommi等 \\
%  	    \hline
%       odom\_alpha1	&  0.2 & 机器人旋转分量中的旋转噪声 \\
%       \hline
%       odom\_alpha2	&  0.2 & 机器人平移分量中的旋转噪声 \\
%       \hline
%       odom\_alpha3	&  0.2 & 机器人平移分量中的平移噪声 \\
%       \hline
%       odom\_alpha4	&  0.2 & 机器人旋转分量中的平移噪声 \\
%  	    \bottomrule[1pt]
%     \end{tabular}
%   \end{table}
% \end{frame}

\begin{frame}{CostFunction}
Ceres提供的CostFunction包括以下三种:

\begin{itemize}
  \item 自动导数(AutoDiffCostFunction):由ceres自行决定导数的计算方式
  \item 数值导数(NumericDiffCostFunction):由用户手动编写导数的数值求解形式,
        通常在残差函数的计算使用无法直接调用的库函数,无法调用AutoDiffCostFunction类构建时使用。
  \item 解析导数(Analytic Derivatives):当导数存在闭合解析形式时使用,用于可基于CostFunciton基类自行编写;
        但由于需要自行管理残差和雅克比矩阵,除非闭合解具有具有明显的精度和效率优势。
\end{itemize}
\end{frame}

\begin{frame}[fragile]
  
  \frametitle{AutoDiffCostFunction}

  以AutoDiffCostFunction为例说明,其构造函数为:

   % \begin{lstlisting}[title=Myfile, frame=shadowbox]
  \begin{lstlisting}[frame=shadowbox]
    ceres::AutoDiffCostFunction<CostFunctor, int residualDim, int paramDim> (CostFunctor* functor);
  \end{lstlisting}

  \begin{itemize}
    \item 模板参数依次为仿函数(functor)类型CostFunctor,残差维数residualDim和参数维数paramDim,接受参数类型为仿函数指针CostFunctor*。
    \item 仿函数CostFunctor:仿函数的本质为结构体struct或者类class,由于重载了()运算符,使得其能够具有和函数一样的调用行为,因此被称为仿函数。
  \end{itemize}
  

\end{frame}

\begin{frame}[fragile]
  \frametitle{迭代器分类标签}
  \begin{lstlisting}[title= carto/mapping/2d/xy\_index.h, frame=shadowbox]
    class XYIndexRangeIterator: public std::iterator<std::input_iterator_tag, Eigen::Array2i>
  \end{lstlisting}
  五种迭代器分类:

  \begin{itemize}
    \item input\_iterator\_tag 对应 InputIterator
    \item output\_iterator\_tag 对应 OutputIterator
    \item forward\_iterator\_tag 对应 ForwardIterator
    \item bidirectional\_iterator\_tag 对应 BidirectionalIterator
    \item random\_access\_iterator\_tag 对应 RandomAccessIterator
  \end{itemize}

  迭代器分类标签可以用以标示某个迭代器的分类,可以根据这一分类所要求的特性来选择最优算法。
\end{frame}

\begin{frame}
  \frametitle{std::lower\_bound, std::upper\_bound}
  \begin{itemize}
    \item std::lower\_bound(first,last,val) 在first和last中的前闭后开区间进行二分查找,返回大于或等于val的第一个元素位置。
      如果所有元素都小于val,则返回last的位置。
    \item std::upper\_bound(first, last, val) 返回的在前闭后开区间查找的关键字的上界,返回大于val的第一个元素位置。
    \item std::binary\_search(first, last, val) 返回的是是否存在这么一个数,是一个bool值。
  \end{itemize}
\end{frame}

\begin{frame}
  \frametitle{tuple}
  tuple是一个固定大小的不同类型值的集合。
  \begin{itemize}
    \item std::tie 它会创建一个元组的左值引用同时可以解包tuple。
    \item std::ignore 通过占位符来表示不解某个位置的值。
    \item std::forward\_as\_tuple 创建右值的引用元组。
    \item std::tuple\_element  获取元素类型。
  \end{itemize}
  
\end{frame}